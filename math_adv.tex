\documentclass[12pt]{article}
\usepackage[utf8]{inputenc}
\usepackage{amsmath, amssymb, amsthm}
\usepackage{enumitem}
\usepackage{geometry}
\usepackage{fancyhdr}
\usepackage{mathtools}
\geometry{margin=1in}

\pagestyle{fancy}
\fancyhf{}
\lhead{MATH 145 / MATH 147 Advanced Practice}
\rfoot{Page \thepage}

\newtheorem{problemx}{Problem}
\newenvironment{problem}[1]{%
	\begin{problemx}[#1]\leavevmode\\[0.5em] % linebreak and small vertical space
	}{%
	\end{problemx}
}

% Solution environment with qed box at the end and paragraph spacing
\newenvironment{solution}{%
	\par\medskip
	\noindent\textbf{Solution.}\par\nopagebreak
}{%
	\hfill \qed \par\medskip
}

\title{Advanced Practice Problems \\ \large MATH 145 / MATH 147}
\author{bwL3, sageman1134, antonahill, and longing5930}
\date{}

\begin{document}
	
	\maketitle
	
	\begin{problem}[1]
		We say that a set $X$ of real numbers is \emph{bounded above} if there exists a real number $\alpha$ such that $x \leq \alpha$ for all $x \in X$. Similarly, $X$ is \emph{bounded below} if there exists a real number $\beta$ such that $x \geq \beta$ for all $x \in X$.
		
		\vspace{1em}
		
		Let $S$ be the set of all real numbers satisfying
		\[
		x^3 + 2x < 4.
		\]
		Is the set bounded above? Is it bounded below? Prove your answers.
	\end{problem}
	
	\begin{solution}
		We can define $S$ as
		\[
		S = \{ x \in \mathbb{R} \mid x^3 + 2x - 4 < 0 \}.
		\]
		
		Let 
		\[
		g(x) = x^3 + 2x - 4.
		\]
		
		Suppose there exists some $r$ such that $g(r) = 0$.
		
		We check if $g$ is strictly increasing by considering
		\[
		g(x) - g(y) = (x^3 + 2x) - (y^3 + 2y) = (x - y)(x^2 + xy + y^2) + 2(x - y) = (x - y)(x^2 + xy + y^2 + 2).
		\]
		Because for any real $x, y$, the term $x^2 + xy + y^2 + 2$ is always positive, and if $x > y$ then $(x - y) > 0$, it follows that
		\[
		g(x) - g(y) > 0 \quad \text{whenever } x > y.
		\]
		
		This shows $g$ is strictly increasing.
		
		Since $g$ is strictly increasing and $g(r) = 0$, it follows that
		\[
		g(x) < 0 \quad \text{for } x < r, \quad \text{and} \quad g(x) > 0 \quad \text{for } x > r.
		\]
		
		Thus,
		\[
		S = \{ x \in \mathbb{R} : x < r \}.
		\]
		
		Therefore, $S$ is bounded above by $r$, but since it contains arbitrarily large negative numbers, it is not bounded below.
	\end{solution}
	
\begin{problem}[2]
	
	The complex numbers $\mathbb{C}$ are a number system which, as a set, is the set of all ordered pairs of real numbers $\mathbb{C} = \{(a,b) \mid a,b \in \mathbb{R}\}$. The ordered pair $(a,b)$ is usually written $a + bi$. Complex numbers can be added, subtracted, multiplied, and divided as follows.
	
	\begin{itemize}[leftmargin=*]
		\item Addition:
		\[
		(a + bi) + (c + di) = (a+c) + (b+d)i.
		\]
		\item Subtraction:
		\[
		(a + bi) - (c + di) = (a-c) + (b-d)i.
		\]
		\item Multiplication:
		\[
		(a + bi)(c + di) = (ac - bd) + (ad + bc)i.
		\]
		\item Division (for $c + di \neq 0$):
		\[
		\frac{a + bi}{c + di} = \frac{(a + bi)(c - di)}{(c + di)(c - di)} = \frac{(ac + bd) + (bc - ad)i}{c^2 + d^2}.
		\]
	\end{itemize}
	
	Prove that there is no order relation "$<$" on $\mathbb{C}$ that satisfies the five order axioms:
	
	\begin{enumerate}[leftmargin=*]
		\item For any $a,b \in \mathbb{C}$, exactly one of the following holds: $a < b$, $a = b$, or $b < a$.
		\item For any $a,b,c \in \mathbb{C}$, if $a < b$ and $b < c$, then $a < c$.
		\item For any $a,b,c \in \mathbb{C}$, if $a < b$, then $a + c < b + c$.
		\item For any $a,b,c \in \mathbb{C}$, if $a < b$ and $0 < c$, then $ac < bc$.
		\item For any $a,b,c \in \mathbb{C}$, if $a < b$ and $c < 0$, then $bc < ac$.
	\end{enumerate}
	
\end{problem}

\begin{solution}
	We proceed by contradiction.
	
	Suppose there exists an order relation "$<$" on $\mathbb{C}$ satisfying the five order axioms above, making $(\mathbb{C}, +, \cdot, <)$ an ordered field.
	
	Consider the imaginary unit $i \in \mathbb{C}$ where $i^2 = -1$.
	
	By axiom (1), exactly one of the following must hold:
	\[
	i > 0, \quad i = 0, \quad \text{or} \quad i < 0.
	\]
	
	We know $i \neq 0$, so either $i > 0$ or $i < 0$.
	
	\textbf{Case 1:} Suppose $i > 0$.
	
	Then by axiom (4), let $a = 0$, $b = i$, $c = i$
	\[
	0 \cdot i < i \cdot i.
	\]
	
	This implies $0 < -1$, which is a contradiction.
	
	\textbf{Case 2:} Suppose $i < 0$.
	
	Then $-i > 0$ and by the same reasoning
	\[
	0 < (-i)^2 = (-1)^2(i)^2 = -1
	\]
	
	In both cases, we conclude
	\[
	0 < -1.
	\]
	
	However, in any ordered field, $1 > 0$ (because $1 = 1$ and is the multiplicative identity).
	
	Adding inequalities preserves order by axiom (3), so
	\[
	-1 > 0 \implies (-1) + 1 > 0 + 1 \implies 0 > 1,
	\]
	which contradicts the fact that $1 > 0$.
	
	Therefore, our assumption that such an order relation exists on $\mathbb{C}$ leads to a contradiction.
	
	\medskip
	
	\noindent\textbf{Conclusion:} There is no order relation "$<$" on the complex numbers $\mathbb{C}$ satisfying all five axioms of an ordered field.
\end{solution}

	
	\begin{problem}[3]
\textbf{Q3.} Given two sets, $A$ and $B$, the \textit{Cartesian product} $A \times B$ is defined as the set of all ordered tuples $(a, b)$, where $a \in A$ and $b \in B$. Using the set notation, this can be written as

$$A \times B = \{ (a, b) \mid a \in A, b \in B \}$$

Let $A$ and $B$ be sets. A \textit{function} from $A$ to $B$, denoted by $f : A \rightarrow B$, is a subset $f$ of $A \times B$ such that for all $a \in A$ there exists a unique $b \in B$ such that $(a, b) \in f$. We write $b = f(a)$ and call the $b$ the \textit{image of} $a$ under $f$. We say $f : A \rightarrow B$ is \textit{injective} if for all $b \in B$ there exists at most one $a \in A$ such that $f(a) = b$. We say $f : A \rightarrow B$ is \textit{surjective} if $\forall b \in B$, $\exists a \in A$ such that $f(a) = b$. Finally, $f : A \rightarrow B$ is \textit{bijective} if it is both injective and surjective. Determine if the following functions are injective and/or surjective.

\begin{enumerate}
	\item $f : \mathbb{Z} \times \mathbb{Z} \rightarrow \mathbb{Q}$, $f(x, y) = \dfrac{x + 1}{x^2 + y^2 + 2}$
	
	\item $f : \mathbb{R} \times \mathbb{R} \rightarrow \mathbb{R}$, $f(x, y) = xy$
	
	\item $f : \mathbb{R} \rightarrow (0, 1)$, $f(x) = \dfrac{x^2}{1 + x^2}$
\end{enumerate}
	\end{problem}
	
	\begin{solution}
		
		\textbf{1.}
		
\textbf{(a) Injective}

We want to check if the function is injective. The function $f$ is injective if whenever $f(x_1, y_1) = f(x_2, y_2)$, it implies that $(x_1, y_1) = (x_2, y_2)$. 

$$\text{Suppose } f(x_1, y_1) = f(x_2, y_2)$$
Then, we have:
$$\frac{x_1 + 1}{x_1^2 + y_1^2 + 2} = \frac{x_2 + 1}{x_2^2 + y_2^2 + 2}$$
This does not necessarily imply that $(x_1, y_1) = (x_2, y_2)$.

\textit{Counterexample:}
Let $f(0, 0) = \frac{1}{2}$ and $f(1, 1) = \frac{1}{2}$, but $(0, 0) \neq (1, 1)$.

Thus, the function is \textbf{not injective}.

\textbf{Surjective}

We need to check if every rational number is attainable.

$$f(x,y) = \frac{x+1}{x^2 + y^2 + 2}, x,y \in \mathbb{Z}$$

Let $f(x,y) = -1$. $-1$ can be written as $-k/k$ or $k/-k$ for all $k$

\textbf{Case 1: } 

$$\frac{x+1}{x^2+y^2+2} = k/-k$$

But, $x^2 + y^2 + 2$ is strictly positive for all $x,y \in \mathbb{Z}$

\textbf{Case 2: }

$$\frac{x+1}{x^2+y^2+2} = -k/k$$

We get the following system of equations.

$$x+ 1 = -k \ \text{and} \ x^2 + y^2 + 2 = k$$

From \( x + 1 = -k \), we get \( x = -(k + 1) \). Substituting into the second equation:

\[
x^2 + y^2 + 2 = k \quad \Rightarrow \quad (-(k + 1))^2 + y^2 + 2 = k
\]

\[
(k + 1)^2 + y^2 + 2 = k \quad \Rightarrow \quad k^2 + 2k + 1 + y^2 + 2 = k
\]

\[
k^2 + k + 3 + y^2 = 0
\]

\[
k^2 + y^2 + 3 = -k
\]

Since the left side, $k^2 + y^2 + 3$ is always positive, and the right side is negative

we reach a contradiction. That is, a positive number cannot equal a negative number.

Hence, no such integers \(x, y\) exist for which \(f(x, y) = -1\). Therefore, \(-1 \notin \text{Im}(f)\), so \(f\) is not surjective.

\medskip

\noindent\textbf{Conclusion:} The function \(f : \mathbb{Z} \times \mathbb{Z} \to \mathbb{Q}\), defined by \(f(x, y) = \dfrac{x + 1}{x^2 + y^2 + 2}\), is neither injective nor surjective.

\textbf{2.}

Consider the function \( f : \mathbb{R} \times \mathbb{R} \rightarrow \mathbb{R} \) given by \( f(x, y) = xy \).

\textbf{Injective:}

Suppose \( f(x_1, y_1) = f(x_2, y_2) \), i.e.,
\[
x_1 y_1 = x_2 y_2.
\]
This does not imply \( (x_1, y_1) = (x_2, y_2) \).

\textit{Counterexample:} Let \( (x_1, y_1) = (2, 3) \) and \( (x_2, y_2) = (1, 6) \). Then
\[
f(2, 3) = 6 = f(1, 6),
\]
but \( (2, 3) \neq (1, 6) \). So the function is \textbf{not injective}.

\textbf{Surjective:}

Let \( r \in \mathbb{R} \). We want to find \( (x, y) \in \mathbb{R} \times \mathbb{R} \) such that \( xy = r \).

\textit{Case 1:} If \( r = 0 \), then \( f(0, y) = 0 \) for any \( y \), so 0 is in the image.

\textit{Case 2:} If \( r \neq 0 \), let \( x = r \), \( y = 1 \). Then \( f(x, y) = r \cdot 1 = r \).

Hence, every real number has a preimage, so the function is \textbf{surjective}.

\medskip

\noindent\textbf{Conclusion:} The function \( f(x, y) = xy \) is \textbf{not injective} but \textbf{surjective}.

\bigskip

\textbf{3.}

Consider the function \( f : \mathbb{R} \rightarrow (0, 1) \) defined by
\[
f(x) = \frac{x^2}{1 + x^2}.
\]

\textbf{Injective:}

Suppose \( f(x_1) = f(x_2) \). Then
\[
\frac{x_1^2}{1 + x_1^2} = \frac{x_2^2}{1 + x_2^2}.
\]
Cross-multiplying:
\[
x_1^2 (1 + x_2^2) = x_2^2 (1 + x_1^2).
\]
Expanding both sides:
\[
x_1^2 + x_1^2 x_2^2 = x_2^2 + x_1^2 x_2^2.
\]
Subtracting \( x_1^2 x_2^2 \) from both sides:
\[
x_1^2 = x_2^2 \quad \Rightarrow \quad x_1 = \pm x_2.
\]
So the function maps both \( x \) and \( -x \) to the same value. For example:
\[
f(1) = \frac{1}{2} = f(-1),
\]
but \( 1 \neq -1 \). Therefore, the function is \textbf{not injective}.

\textbf{Surjective:}

Let \( y \in (0, 1) \). We want to find \( x \in \mathbb{R} \) such that
\[
\frac{x^2}{1 + x^2} = y.
\]
Multiply both sides by \( 1 + x^2 \):
\[
x^2 = y(1 + x^2) \quad \Rightarrow \quad x^2 = y + yx^2.
\]
Rewriting:
\[
x^2 - yx^2 = y \quad \Rightarrow \quad x^2(1 - y) = y \quad \Rightarrow \quad x^2 = \frac{y}{1 - y}.
\]
Since \( y \in (0, 1) \), the right-hand side is positive. So such an \( x \) exists, namely \( x = \pm \sqrt{\frac{y}{1 - y}} \in \mathbb{R} \).

Hence, the function is \textbf{surjective}.

\medskip

\noindent\textbf{Conclusion:} The function \( f(x) = \frac{x^2}{1 + x^2} \) is \textbf{not injective} but \textbf{surjective}.


	\end{solution}
	
	\begin{problem}[4]
		\textbf{Q4.} Let $X$ be a set and let $P(X)$ denote the set of all subsets of $X$. Prove that there does not exist a surjective function $f: X \to P(X)$. (Hint: $A = \{ x \in X : x \notin f(x) \}$.)
	\end{problem}
	
	\begin{solution}
Assume for contradiction that:

$f: X \to P(X)$ is surjective.

For every subset $A \subseteq X$, $\exists a \in X$ such that $f(a) = A$.

Let 
$$A = \{ x \in X \mid x \notin f(x) \}.$$

This is a well-defined subset of $X$, so $A \in P(X)$.

Since $f$ is surjective, $\exists a \in X$ such that $f(a) = A$.

Suppose $a \notin A$. By definition of $A$,

$$a \notin A \Rightarrow a \in f(a).$$

But $f(a) = A$, so 

$$a \in A,$$

a contradiction.

Suppose $a \in A$. Then by definition of $A$,

$$a \in A \Rightarrow a \notin f(a).$$

Since $f(a) = A$, this means 

$$a \notin A,$$

again a contradiction.

Thus, in both cases we get a contradiction. Therefore $f$ is not surjective.
	\end{solution}
	
\begin{problem}[5]
	A nonempty subset $X$ of $\mathbb{Z}$ is said to be \textit{sticky} if for all $x, y \in X$ and $n \in \mathbb{Z}$, $x + y \in X$ and $xn \in X$.
	
	\textbf{(a)} Fix $n \in \mathbb{Z}$. Find the smallest sticky subset of $\mathbb{Z}$ which contains $n$.
	
	\textbf{(b)} Fix $n \in \mathbb{Z}$, different from $\pm1$. Find all sticky subsets $X \subset \mathbb{Z}$ such that (1) $X \neq \mathbb{Z}$; (2) $n \in X$; (3) whenever $X \subset Y$ which is a sticky subset of $\mathbb{Z}$, then $X = Y$. How many sticky subsets can you find? 
\end{problem}

\begin{solution}
	
	\textbf{(a)}
	
	Given $n \in \mathbb{Z}$, define 
	\[
	X_n = \{ k n \mid k \in \mathbb{Z} \} = n \mathbb{Z}.
	\]
	
	We claim that $X_n$ is the smallest sticky subset of $\mathbb{Z}$ containing $n$.
	
	\textit{Proof that $X_n$ is sticky:}
	
	Let $x = a n$ and $y = b n$ be arbitrary elements of $X_n$ with $a, b \in \mathbb{Z}$.
	
	Then:
	\[
	x + y = a n + b n = (a + b) n \in X_n
	\]
	and for any $m \in \mathbb{Z}$:
	\[
	x m = a n m = (a m) n \in X_n.
	\]
	
	Thus, $X_n$ is closed under addition and multiplication by any integer, so $X_n$ is sticky.
	
	\textit{Minimality of $X_n$:}
	
	Suppose $Y$ is any sticky subset of $\mathbb{Z}$ containing $n$. Then:
	
	\begin{itemize}
		\item Since $n \in Y$ and $Y$ is sticky, for any integer $k$, by repeatedly adding $n$ to itself (closure under addition), we get $k n \in Y$.
		\item Also, closure under multiplication by any integer implies all integer multiples of $n$ are in $Y$.
	\end{itemize}
	
	Hence $n \mathbb{Z} \subseteq Y$.
	
	Therefore, the smallest sticky subset of $\mathbb{Z}$ containing $n$ is $n \mathbb{Z}$.
	
\textbf{(b)}

Fix $n \in \mathbb{Z}$ with $n \ne \pm 1$.

We are asked to find all sticky subsets $X \subset \mathbb{Z}$ satisfying:

\begin{enumerate}[label=(\roman*)]
	\item $X \ne \mathbb{Z}$,
	\item $n \in X$,
	\item $X$ is maximal among all proper sticky subsets of $\mathbb{Z}$ containing $n$.
\end{enumerate}

\textit{Recall:} A subset $X \subset \mathbb{Z}$ is sticky if it is closed under:
\begin{itemize}
	\item addition: if $x, y \in X$ then $x + y \in X$, and
	\item integer multiplication: if $x \in X$, $m \in \mathbb{Z}$ then $mx \in X$.
\end{itemize}

As shown in part (a), the smallest sticky subset of $\mathbb{Z}$ containing $n$ is:
\[
X_n = n\mathbb{Z} = \{ kn \mid k \in \mathbb{Z} \}.
\]
More generally, any sticky subset of $\mathbb{Z}$ is of the form $d\mathbb{Z}$ for some $d \in \mathbb{N}$.

We want all proper sticky subsets $X = d\mathbb{Z} \subsetneq \mathbb{Z}$ such that:
\begin{itemize}
	\item $n \in d\mathbb{Z} \iff d \mid n$,
	\item and $d\mathbb{Z}$ is maximal among such proper sticky sets.
\end{itemize}

So we want all divisors $d > 1$ of $n$ such that $d\mathbb{Z}$ is not properly contained in any other proper sticky subset that contains $n$.

But since $d_1\mathbb{Z} \subset d_2\mathbb{Z} \iff d_2 \mid d_1$, the \emph{maximal} proper sticky subsets containing $n$ are the ones where $d$ is the smallest possible positive divisor of $n$ greater than $1$—but in fact, every $d > 1$ dividing $n$ gives a distinct such sticky set $d\mathbb{Z}$, and they are all maximal under inclusion among proper sticky sets containing $n$.

\textbf{Conclusion:} The sticky subsets $X \subsetneq \mathbb{Z}$ satisfying the conditions are precisely the sets:
\[
X = d\mathbb{Z} \quad \text{where } d > 1 \text{ and } d \mid n.
\]

\noindent The number of such subsets is equal to the number of positive divisors of $n$ greater than $1$.

	
\end{solution}

	\begin{problem}[6]
\textbf{Q6.} Suppose that for each natural number $n$, we make a statement $P(n)$ (which can either be true or false). The \textit{Principle of Mathematical Induction} tells us that:
\begin{enumerate}
	\item If $P(1)$ is true, and
	\item If for any natural number $k$, $P(k)$ being true implies that $P(k + 1)$ is also true,
\end{enumerate}
then $P(n)$ is true for all $n \in \mathbb{N}$.

Find the flaw in the following proof by mathematical induction, which seems to suggest that all real numbers are equal.

Let $P(n)$ be the statement that in any set of real numbers $\{x_1, x_2, \dots, x_n\}$, all of the numbers have the same value. 

It is clear that $P(1)$ is true. Now assume that $P(k)$ is true for some natural number $k$. Consider any set of $(k+1)$ real numbers $\{x_1, x_2, \dots, x_{k+1}\}$. 

If we remove $x_1$, then we have a set of $k$ numbers, so by $P(k)$, we have that 
$$x_2 = x_3 = \dots = x_{k+1}.$$

If, instead, we remove $x_k$, then we again have a set of $k$ numbers, so 
$$x_1 = x_2 = \dots = x_{k-1} = x_k.$$

Combining these two results, we see that 
$$x_1 = x_2 = x_3 = \dots = x_{k+1},$$
proving that $P(k+1)$ is true. By the Principle of Mathematical Induction, $P(n)$ is true for all natural numbers $n$.
	\end{problem}
	
\begin{solution}
	We are asked to identify the flaw in a proof that falsely claims all real numbers are equal, using mathematical induction on the following statement:
	
	\begin{center}
		$P(n)$: "In any set of $n$ real numbers, all the numbers are equal."
	\end{center}
	
	\textbf{Base Case ($n = 1$):}  
	The statement $P(1)$ is true. Any set with a single real number trivially has all elements equal.
	
	\textbf{Inductive Step:}  
	The proof assumes $P(k)$ is true for some $k \in \mathbb{N}$ and attempts to prove $P(k+1)$. The idea is to take a set of $k+1$ real numbers $\{x_1, x_2, \dots, x_{k+1}\}$ and consider two overlapping subsets of size $k$:
	
	\begin{itemize}
		\item Removing $x_1$ gives the set $\{x_2, x_3, \dots, x_{k+1}\}$. By the inductive hypothesis, all these values are equal:  
		\[
		x_2 = x_3 = \dots = x_{k+1}.
		\]
		
		\item Removing $x_{k+1}$ gives the set $\{x_1, x_2, \dots, x_k\}$. Again by the inductive hypothesis,  
		\[
		x_1 = x_2 = \dots = x_k.
		\]
	\end{itemize}
	
	Since both subsets include $x_2$, the proof claims this common value links the two equalities and implies
	\[
	x_1 = x_2 = \dots = x_{k+1},
	\]
	and thus concludes that $P(k+1)$ is true.
	
	\textbf{Flaw in the Argument:}
	
	This reasoning appears valid for \( k \geq 2 \), but fails when \( k = 1 \), which is exactly the first step of the induction.
	
	To illustrate the failure:
	
	\begin{itemize}
		\item Let \( k = 1 \). Then we try to prove \( P(2) \): "Any set of two real numbers are equal."
		\item The set is \( \{x_1, x_2\} \).
		\item Removing \( x_1 \) leaves \( \{x_2\} \), which trivially satisfies \( P(1) \), so no new information is obtained.
		\item Removing \( x_2 \) leaves \( \{x_1\} \), which again trivially satisfies \( P(1) \).
	\end{itemize}
	
	However, since these two subsets have no elements in common (they are disjoint), we cannot compare or "link" their values. There's no overlapping element (like \( x_2 \) in the general case) to ensure consistency between the values.
	
	As a result, we cannot conclude that \( x_1 = x_2 \). This breaks the inductive step at \( k = 1 \), and therefore the entire induction argument collapses.
	
	\textbf{Conclusion:}  
	The flaw lies in the assumption that the two overlapping subsets used in the inductive step always share a common element. This is not true when \( k = 1 \), making the inductive step invalid and the proof incorrect. Hence, the conclusion that all real numbers are equal is false.
\end{solution}

	
	\begin{problem}[7]
\textbf{Q7.} Let $a_n > 0$ be a positive real number for each natural number $n$ and define $S_n = a_1 + a_2 + \dots + a_n$. It is clear that $0 \leq S_n \leq S$, for some $S \in \mathbb{R}$.

Find examples of $a_n$ satisfying the following conditions:

\textbf{(a)} The set $\{S_n\}$ is not bounded above, i.e., for any real number $M$, there exists $n \in \mathbb{N}$ such that $S_n > M$.

\textbf{(b)} The set $\{S_n\}$ is bounded above, i.e., there exists a fixed number $S$ such that $S_n \leq S$ for all $n \in \mathbb{N}$.
	\end{problem}
	
	\begin{solution}
		
		\textbf{(a)} We need to show that the sequence of partial sums $S_n$ diverges to infinity.

Let $a_n = 1$ for all $n$. Then, the partial sum is given by
$$S_n = \sum_{k=1}^{n} 1 = n.$$
This sequence grows without bound:
$$S_1 = 1, \quad S_2 = 2, \quad S_3 = 3, \quad \dots, \quad S_n = n \quad \text{as} \quad n \to \infty.$$
Thus, this sequence is not bounded above.

\textbf{(b)} The set $\{S_n\}$ is bounded above, i.e., there exists a fixed number $S$ such that $S_n \leq S$ for all $n \in \mathbb{N}$.

\textit{Solution:} We need the infinite series to converge. 

Let $a_n = \frac{1}{2^n}$ for all $n$. Then, the partial sum is:
$$S_n = \sum_{k=1}^{n} \frac{1}{2^k}.$$
The infinite series is given by
$$\sum_{k=1}^{\infty} \frac{1}{2^k} = 1.$$
This series converges to 1, so $S_n \leq 1$ for all $n$, and $S_n$ converges to 1.
	\end{solution}
	
\end{document}
